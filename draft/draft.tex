\documentclass[times, 10pt, twocolumn]{article}
\usepackage{latex8}
\usepackage[dvips]{graphicx}
\usepackage[fleqn]{amsmath}
\usepackage{amsthm}
\usepackage{txfonts}
\usepackage{courier}
\usepackage{subfigure}
\usepackage{comment}
\usepackage{url}
\usepackage[square,numbers]{natbib}

%-----------------------------------------
% need for camera-ready
\pagestyle{empty}

%----------------------------------------

\begin{document}

\title{
Toward Adaptive GPU Resource Management for Embedded Real-Time Systems
}

\author {
Junsung Kim and Ragunathan (Raj) Rajkumar\\
\\
Department of Electrical and Computer Engineering\\
Carnegie Mellon University
\and
Shinpei Kato and Scott Brandt\\
\\
Department of Computer Science\\
University of California, Santa Cruz
}

\maketitle

%-----------------------------------------
% need for camera-ready
\thispagestyle{empty}

\begin{abstract}
 In this paper, we present conceptual frameworks for GPU applications to
 adjust their task execution times upon given workloads.
\end{abstract}

\section{Introduction}
\label{sec:introduction}

\section{System Model}
\label{sec:system_model}

We assume such embedded systems that equip GPUs as compute devices.
The application task starts execution on the CPU, and offloads its
data-parallel compute-intensive workload onto the GPU when needed.
Once offloaded onto the GPU, the task becomes non-preemptive due to many
reasons.
In fact, it is technically possible to preempt the running task on the
GPU by loading and restoring its context, but it requires additional
firmware, runtime, and OS support, and the preemption cost would be
non-trivial due to a very large set of GPU registers and states.
We hence restrict our attention to a non-preemptive execution model for
GPU computing.
GPUs may also pose some constraints in multi-tasking.
Even the NVIDIA Fermi architecture~\cite{Fermi}, one of the most popular
GPU product lines, allows only one context to use GPU resources at
once, though this context may spawn multiple GPU kernels (jobs)
simultaneously.
In other words, if task-level parallelism is required, the entire system
must run in the same context.
We however believe that this constraint will not limit the concept of
this paper.
We can use the same context to exploit concurrent parallel job
executions, and it should be enough to demonstrate a proof of concept. 
We also expect that future product lines will remove this concern.

We consider real-time applications where each task runs in a periodic or
sporadic manner under deadline constraints.
Such a task set and application includes motion planning and 
vision-based perception in state-of-the-art autonomous driving
vehicles.
We also presume the computing demand of each task to be highly
variable.
For example, planning and perception workloads are governed by the
number of objects, the size of data, and the desired quality of output.
These workloads are also well parallelizable using the GPU.
Hence, the autonomous driving tasks and applications are our primary
target, but the contribution of this paper is not limited to those but
is also generally applicable to highly vaiable workloads running on the
GPU.

\section{Adaptive GPU Resource Management}
\label{sec:adaptivity_support}

In this section, we provide an idea of adaptivity support for GPU
applications.
We particularly focus on computing resource allocation problems.
It is true that some embedded real-time systems exhibit highly variable
workloads.
Since GPUs integrate a great number of compute cores on a chip, we can
support even highly variable workloads in a timely manner by adjusting
allocated cores at runtime.

There are several levels of approaches for adaptive GPU resource
management.
Previous work~\cite{Kato_RTAS11, Kato_RTSS11, Kato_ATC11} took
time-driven approaches that control timings and the amounts of time
allowed to access GPU resources, \textit{i.e.}, scheduling and
reservation.
In these time-driven approaches, application tasks need not to be aware
of what is happening in GPU resource management, because it is handled
by the OS or runtime scheduler.
However, they can never manage task execution times.
They also limit the number of contexts that can access the GPU
simultaneously to remove performance interference.
Therefore, GPU resources could be wasted if a running context does not
fully use compute cores.

We consider a different approach than previous work that enables GPU
applications to adjust their task execution times upon workload.
We believe that the number of cores used in the program is a major
factor that could change the execution time, and hence explore how to
adjust it at runtime.
It is important to note that programmers are typically responsible for
allocating the number of cores (or threads mapped to cores) in GPU
programming.
In order to adjust the number of cores at runtime, it is essential to
provide the programmers with an interface to obtain the information on
the number of cores available or allocated for the program.
In the following, we present two frameworks that could achieve our idea.
Although they are still at an idea stage, we plan to implement a real
system and demonstrate its capability, leveraging open-source
software~\cite{Kato_OSPERT11}.

\subsection{Explicit Adjustment}

At the end of each period, the programmer calls a API function provided
by our framework that returns the latest task execution time.
The programmer next calls either of the following two API functions.
One increases the number of cores to be used in the next period or GPU
invocation to speed up the program.
The other decreases it to speed down the program.
This framework would work fine because the programmer often knows the
desired task execution time to meet the frame-rate or deadline.
It is also flexible in that the programmer can determine when to
increase or decrease the number of cores.

\subsection{Implicit Adjustment}

The programmer specifies the desired task execution time as a set point
before the task starts.
When the task uses the GPU, the runtime system consistently updates the
number of cores available for the corresponding task in the next period
based on the previous execution time records.
It is a programmer's duty to check the number of available cores before
offloading the computation onto the GPU.


\section{Summary}
\label{sec:summary}

%\setlength{\bibsep}{2.5pt}
%\renewcommand{\baselinestretch}{0.94}
\bibliographystyle{plain}
{\footnotesize
\bibliography{references}
}

\end{document}
